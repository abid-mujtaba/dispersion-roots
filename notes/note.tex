\documentclass[12pt,a4paper]{article}

\usepackage[margin=0.8in]{geometry}
\usepackage[parfill]{parskip}
\usepackage{amsmath}
\usepackage{amsfonts}

\begin{document}
    The roots of the dispersion relation $\epsilon(k_\perp, \omega)$ \textbf{CITE dispersion relation} have been calculated for various values of $\Lambda_j$, $\kappa_j$, $\frac{n0_h}{n0_e}$, and $\frac{T_h}{T_c}$.
    The resulting plots of the roots of the dispersion relation, while differing from each other, do share some common features.
    All of the plots are split up in to bands with boundaries at integer values of $\frac{\omega}{\omega_{ce}}$.
    In each band all of the plots asymptotically approach the lower boundary as $k_\perp \rightarrow \infty$.

    The structure of the plots can be explained by analyzing the dispersion relation analytically.
    A better form, both for analytical analysis and numerical calculation (especially at $k_\perp = 0$), can be derived by using the series definition of the generalized hypergeometric function \textbf{CITE http://dlmf.nist.gov/16.2}.

    \begin{align}
        \nonumber \epsilon(k_\perp, \omega) = 1 +& \sum_j \frac{4 \, (\kappa_j - \frac{3}{2}) \, v^2_{th}}{\omega^2_{ce} \, \lambda^2_{\nu c j}} \times \sum_{n = 1}^3 \alpha_n \left( \sum_{m = 1}^\infty \frac{ -\, (\frac{1}{2})_m \, (n)_m }{(n - \frac{1}{2} - \kappa_j)_m \, (1 - \frac{\omega}{\omega_{cj}})_m (1 + \frac{\omega}{\omega_{cj}})_m } \frac{(2 \lambda_j)^{(m - 1)}}{m!} \right. \\
        %
        \nonumber -& \left. \frac{(\kappa_j + \frac{1}{2})_{-n}}{(n - 1)!} \, \pi^{\frac{1}{2}} \csc\left(\pi \frac{\omega}{\omega_{cj}}\right) \left(\frac{\omega}{\omega_{cj}}\right) \Gamma(\kappa_j + 2 - n) \Gamma(n - \frac{3}{2} - \kappa_j) \right.\\
        %
        \times& \left. \sum_{m = 0}^\infty \frac{(\kappa_j - n + 2)_m \, (\kappa_j + \frac{3}{2})_m}{(\kappa_j + \frac{5}{2} - n)_m \, \Gamma(\kappa - n + \frac{5}{2} - \frac{\omega}{\omega_{cj}}) \, \Gamma(\kappa_j - n + \frac{5}{2} + \frac{\omega}{\omega_{cj}})} \frac{(2 \lambda_j)^{(\kappa_j + \frac{1}{2} - n + m)}}{m!} \right)
    \end{align}

    The $(1 - \frac{\omega}{\omega_{cj}})_m$, pochammer symbol in the denominator goes to zero whenever $\frac{\omega}{\omega_{cj}} \in \mathbb{Z}^+$ (and $m > \frac{\omega}{\omega_{cj}}$) creating a pole.
    Other than the set $\{(k_\perp, \omega) : k_\perp > 0, \frac{\omega}{\omega_{cj}} \in \mathbb{Z}^+\}$ the dispersion relation is continuous (with the possible exception of $k_\perp = 0$ discussed below).
    It is these ``horizontal lines'' of discontinuity that give rise to the horizontal bands in the graph of $\epsilon(k_\perp, \omega) = 0$.

    The poles at integer values of $\frac{\omega}{\omega_{cj}}$ behave like $(1 - \frac{\omega}{\omega_{cj}} + m)^{-1}$ (for $m \in \mathbb{Z}^+$) which means $\epsilon(k_\perp, \omega)$ changes sign at each boundary as $\omega$ increases.
    In each band as $\omega$ increases (for constant $k_\perp$) the dispersion relation starts off at $-\infty$ on the lower boundary and ends up at $+\infty$ on the upper boundary.
    Continuity and the Bolzano's Theorem (Intermediate Value Theorem \textbf{CITE}) then guarantees that the dispersion relation will definitely have a zero (root) in the middle, which is why we observe a plot of roots in every band.

    % REFACTOR Below

    $k_\perp = 0$ has no physical significance but it ends up controlling the shape of the plot in each band. To study the dispersion relation at $k_\perp = 0$ we will have to use its series definition to refactor the expression to (the same form that was used for the numerical calculations):

    \begin{align}
        \epsilon(k_\perp, \omega) = 1 +& \sum_j \frac{4 \, (\kappa_j - \frac{3}{2}) \, v^2_{th}}{\omega^2_{ce} \, \lambda^2_{\nu c j}} \times \sum_{n = 1}^3 \alpha_n \left( \sum_{m = 1}^\infty \frac{ -\, (\frac{1}{2})_m \, (n)_m }{(n - \frac{1}{2} - \kappa_j)_m \, (1 - \frac{\omega}{\omega_{cj}})_m (1 + \frac{\omega}{\omega_{cj}})_m } \frac{(2 \lambda_j)^{(m - 1)}}{m!} \right. \\
        %
        -& \left. \frac{(\kappa_j + \frac{1}{2})_{-n}}{(n - 1)!} \, \pi^{\frac{1}{2}} \csc\left(\pi \frac{\omega}{\omega_{cj}}\right) \left(\frac{\omega}{\omega_{cj}}\right) \Gamma(\kappa_j + 2 - n) \Gamma(n - \frac{3}{2} - \kappa_j) \right.\\
        %
        \times& \left. \sum_{m = 0}^\infty \frac{(\kappa_j - n + 2)_m \, (\kappa_j + \frac{3}{2})_m}{(\kappa_j + \frac{5}{2} - n)_m \, \Gamma(\kappa - n + \frac{5}{2} - \frac{\omega}{\omega_{cj}}) \, \Gamma(\kappa_j - n + \frac{5}{2} + \frac{\omega}{\omega_{cj}})} \frac{(2 \lambda_j)^{(\kappa_j + \frac{1}{2} - n + m)}}{m!} \right)
    \end{align}

    In the first category, $\Lambda_j = 0$ or $\kappa_j > \frac{5}{2}$ (for both species) the second term (internal) goes to zero because $\alpha_2 = 0 = \alpha_3$, or the power of $2 \lambda_j$ is positive for all terms, respectively.
    The first term (internal) has only one non-zero term (corresponding to $m = 1$) in the limit $k_\perp \rightarrow 0$, so only a single $\omega^2$ term remains in the expression which means there is only a single root ($\omega_0$) on the $\omega$ axis.
    The dispersion relation is positive below this root and negative above it.
    In the case of \textbf{CITE Henning} one can calculate this root analytically and show that it equals $\frac{\omega_{UH}}{\omega_{cj}}$ which is where the plot touches the $\omega$ axis and the band plots switch from starting at the lower boundary to starting at the upper boundary.

    This behavior can be explained analytically by noting that the dispersion relation is continuous with the sole exception of the segments $\{(k_\perp, \omega): k_\perp = 0, \frac{\omega}{\omega_{cj}} \in \mathbb{Z}^+\}$.
    It follows that $\epsilon(k_\perp = 0, \omega < \omega_0) > 0$ and so continuity demands that all bands for $\omega < \omega_0$ have plots that start at the lower boundary since the positive value of $\epsilon$ in the large $k_\perp$ region must be connected to the positive values on the $\omega$-axis.

    In the band containing the root $\omega_0$ the plot starts at $(0, \omega_0)$. The region below the plot has negative values and the region above it has positive values.

    In the bands above $\omega_0$ since $\epsilon(k_\perp = 0, \omega > \omega_0) < 0$ the plots start at the top boundary since the negative valued $\omega$-axis region has to disconnected from the positive large $k_\perp$ region by the plot itself which shows the points where $\epsilon$ changes sign.

\end{document}
