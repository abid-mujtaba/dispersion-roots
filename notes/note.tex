\documentclass[12pt,a4paper]{article}

\usepackage[margin=0.8in]{geometry}
\usepackage[parfill]{parskip}
\usepackage{amsmath}
\usepackage{amsfonts}

\begin{document}
    The roots of the dispersion relation \textbf{CITE dispersion relation} have been calculated for various values of $\Lambda_j$, $\kappa_j$, $\frac{n0_h}{n0_e}$, and $\frac{T_h}{T_c}$.
    The resulting plots of the roots of the dispersion relation, while differing from each other do share some common features, and can be split in to two broad categories.

    All of the plots are split up in to bands with boundaries at integer values of $\frac{\omega}{\omega_{ce}}$.
    In each band all of the plots asymptotically approach the lower boundary as $k_\perp \rightarrow \infty$.

    The first category corresponds to $\Lambda_j = 0$ or $\kappa_j > \frac{5}{2}$ for both species and one analyzes the bands starting from $\frac{\omega}{\omega_{ce}} = 1$ and going up the initial bands have plots that start (at $k_\perp = 0$) from the lower boundary. At a certain single band in the middle the plot starts in the middle of the band.
    Subsequently for all higher bands the plot starts at the upper boundary.
    Since $\Lambda_j = 0$ corresponds to \textbf{CITE Henning} this is the behavior we see in that case as well.

    This behavior can be explained by analyzing the dispersion relation analytically.
    The band structure arises from the generalized confluent hypergeometric function (with its associated series definition \textbf{CITE http://dlmf.nist.gov/16.2}):

    \begin{equation}
        _2F_3\left(\frac{1}{2}, n; n - \frac{1}{2} - \kappa_j, 1 - \frac{\omega}{\omega_{cj}}, 1 + \frac{\omega}{\omega_{cj}}; 2 \lambda_j \right) = \sum_{m = 0}^{\infty} \frac{(\frac{1}{2})_m \, (n)_m}{(n - \frac{1}{2} - \kappa_j)_m \, (1 - \frac{\omega}{\omega_{cj}})_m \, (1 + \frac{\omega}{\omega_{cj}})_m} \frac{(2 \lambda_j)^m}{m!}
    \end{equation}

    The $(1 - \frac{\omega}{\omega_{cj}})_m$, pochammer symbol in the denominator goes to zero whenever $\omega \in \mathbb{Z}^+$ (and $m$ is large enough) creating a pole in the $_2F_3$.
    Other than these discrete values the dispersion relation is continuous.
    Proving this requires an understanding that the Gamma functions dividing the second $_2F_3$ cancels the poles introducing by the pochammer symbols in its denominator.

    It is this continuity that allows us to ascertain the shape of the plots analytically. The poles at integer values of $\frac{\omega}{\omega_{cj}}$ behave like $\frac{1}{1 - \frac{\omega}{\omega_{cj}} - m}$ for $m \in \mathbb{Z}^+$ which means the dispersion relation changes sign at each boundary as $\omega$ increases. Therefore in each band as $\omega$ increases (for constants $k_\perp$) the dispersion relation starts off at $-\infty$ on the lower boundary and ends up at $+\infty$ on the upper boundary. Continuity then guarantees that dispersion relation will definitely have a zero (root) in the middle.

    $k_\perp = 0$ has no physical significance but it ends up controlling the shape of the plot in each band. To study the dispersion relation at $k_\perp = 0$ we will have to use its series definition to refactor the expression to (the same form that was used for the numerical calculations):

    \begin{align}
        \epsilon(k_\perp, \omega) = 1 +& \sum_j \frac{4 \, (\kappa_j - \frac{3}{2}) \, v^2_{th}}{\omega^2_{ce} \, \lambda^2_{\nu c j}} \times \sum_{n = 1}^3 \alpha_n \left( \sum_{m = 1}^\infty \frac{ -\, (\frac{1}{2})_m \, (n)_m }{(n - \frac{1}{2} - \kappa_j)_m \, (1 - \frac{\omega}{\omega_{cj}})_m (1 + \frac{\omega}{\omega_{cj}})_m } \frac{(2 \lambda_j)^{(m - 1)}}{m!} \right. \\
        %
        -& \left. \frac{(\kappa_j + \frac{1}{2})_{-n}}{(n - 1)!} \, \pi^{\frac{1}{2}} \csc\left(\pi \frac{\omega}{\omega_{cj}}\right) \left(\frac{\omega}{\omega_{cj}}\right) \Gamma(\kappa_j + 2 - n) \Gamma(n - \frac{3}{2} - \kappa_j) \right.\\
        %
        \times& \left. \sum_{m = 0}^\infty \frac{(\kappa_j - n + 2)_m \, (\kappa_j + \frac{3}{2})_m}{(\kappa_j + \frac{5}{2} - n)_m \, \Gamma(\kappa - n + \frac{5}{2} - \frac{\omega}{\omega_{cj}}) \, \Gamma(\kappa_j - n + \frac{5}{2} + \frac{\omega}{\omega_{cj}})} \frac{(2 \lambda_j)^{(\kappa_j + \frac{1}{2} - n + m)}}{m!} \right)
    \end{align}

    In the first category, $\Lambda_j = 0$ or $\kappa_j > \frac{5}{2}$ (for both species) the second term (internal) goes to zero because $\alpha_2 = 0 = \alpha_3$, or the power of $2 \lambda_j$ is positive for all terms, respectively.
    The first term (internal) has only one non-zero term (corresponding to $m = 1$) in the limit $k_\perp \rightarrow 0$, so only a single $\omega^2$ term remains in the expression which means there is only a single root ($\omega_0$) on the $\omega$ axis.
    The dispersion relation is positive below this root and negative above it.
    In the case of \textbf{CITE Henning} one can calculate this root analytically and show that it equals $\frac{\omega_{UH}}{\omega_{cj}}$ which is where the plot touches the $\omega$ axis and the band plots switch from starting at the lower boundary to starting at the upper boundary.

    This behavior can be explained analytically by noting that the dispersion relation is continuous with the sole exception of the segments $\{(k_\perp, \omega): k_\perp = 0, \frac{\omega}{\omega_{cj}} \in \mathbb{Z}^+\}$.
    It follows that $\epsilon(k_\perp = 0, \omega < \omega_0) > 0$ and so continuity demands that all bands for $\omega < \omega_0$ have plots that start at the lower boundary since the positive value of $\epsilon$ in the large $k_\perp$ region must be connected to the positive values on the $\omega$-axis.

    In the band containing the root $\omega_0$ the plot starts at $(0, \omega_0)$. The region below the plot has negative values and the region above it has positive values.

    In the bands above $\omega_0$ since $\epsilon(k_\perp = 0, \omega > \omega_0) < 0$ the plots start at the top boundary since the negative valued $\omega$-axis region has to disconnected from the positive large $k_\perp$ region by the plot itself which shows the points where $\epsilon$ changes sign.

\end{document}
