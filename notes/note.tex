\documentclass[12pt,a4paper]{article}

\usepackage[margin=0.8in]{geometry}
\usepackage[parfill]{parskip}
\usepackage{amsmath}
\usepackage{amsfonts}

\begin{document}
    The roots of the dispersion relation $\epsilon(k_\perp, \omega)$ \textbf{CITE dispersion relation} have been calculated for various values of $\Lambda_j$, $\kappa_j$, $\frac{n0_h}{n0_e}$, and $\frac{T_h}{T_c}$.
    The resulting plots of the roots of the dispersion relation, while differing from each other, do share some common features.
    All of the plots are split up in to bands with boundaries at integer values of $\frac{\omega}{\omega_{ce}}$.
    In each band all of the plots asymptotically approach the lower boundary as $k_\perp \rightarrow \infty$.

    The structure of the plots can be explained by analyzing the dispersion relation analytically.
    A better form, both for analytical analysis and numerical calculation (especially at $k_\perp = 0$), can be derived by using the series definition of the generalized hypergeometric function \textbf{CITE http://dlmf.nist.gov/16.2}.

    \begin{align}
        \nonumber \epsilon(k_\perp, \omega) = 1 -& \sum_j \frac{4 \, (\kappa_j - \frac{3}{2}) \, v^2_{th}}{\omega^2_{cj} \, \lambda^2_{\nu c j}} \times \sum_{n = 1}^3 \alpha_n \left( \sum_{m = 1}^\infty \frac{ (\frac{1}{2})_m \, (n)_m }{(n - \frac{1}{2} - \kappa_j)_m \, (1 - \frac{\omega}{\omega_{cj}})_m (1 + \frac{\omega}{\omega_{cj}})_m } \frac{(2 \lambda_j)^{(m - 1)}}{m!} \right. \\
        %
        \nonumber +& \left. \frac{(\kappa_j + \frac{1}{2})_{-n}}{(n - 1)!} \, \pi^{\frac{1}{2}} \csc\left(\pi \frac{\omega}{\omega_{cj}}\right) \left(\frac{\omega}{\omega_{cj}}\right) \Gamma(\kappa_j + 2 - n) \Gamma(n - \frac{3}{2} - \kappa_j) \right.\\
        %
        \times& \left. \sum_{m = 0}^\infty \frac{(\kappa_j - n + 2)_m \, (\kappa_j + \frac{3}{2})_m}{(\kappa_j + \frac{5}{2} - n)_m \, \Gamma(\kappa_j - n + \frac{5}{2} - \frac{\omega}{\omega_{cj}} + m) \, \Gamma(\kappa_j - n + \frac{5}{2} + \frac{\omega}{\omega_{cj}} + m)} \frac{(2 \lambda_j)^{(\kappa_j + \frac{1}{2} - n + m)}}{m!} \right)
    \end{align}

    The $(1 - \frac{\omega}{\omega_{cj}})_m$, pochammer symbol in the denominator goes to zero whenever $\frac{\omega}{\omega_{cj}} \in \mathbb{Z}^+$ (and $m > \frac{\omega}{\omega_{cj}}$) creating a pole.
    Other than the set $\{(k_\perp, \omega) : k_\perp > 0, \frac{\omega}{\omega_{cj}} \in \mathbb{Z}^+\}$ the dispersion relation is continuous (with the possible exception of $k_\perp = 0$ discussed below).
    It is these ``horizontal lines'' of discontinuity that give rise to the horizontal bands in the graph of $\epsilon(k_\perp, \omega) = 0$.

    The poles at integer values of $\frac{\omega}{\omega_{cj}}$ behave like $(1 - \frac{\omega}{\omega_{cj}} + m)^{-1}$ (for $m \in \mathbb{Z}^+$) which means $\epsilon(k_\perp, \omega)$ changes sign at each boundary as $\omega$ increases.
    In each band as $\omega$ increases (while $k_\perp$ is kept fixed) the dispersion relation starts off at $-\infty$ on the lower boundary and ends up at $+\infty$ on the upper boundary.
    Continuity and the Bolzano's Theorem (Intermediate Value Theorem \textbf{CITE}) then guarantees that the dispersion relation will definitely have a zero (root) in the middle, which is why we observe a plot of roots in every band.

    The plots can be divided in to two broad categories.
    The first category corresponds to $\Lambda_j = 0$ or $\kappa_j > \frac{5}{2}$ (for both species).
    In this category the plots in the lower bands start from the upper boundary and curve down to asymptotically approach the lower boundary.
    In exactly one intermediate band the plot starts in the middle of the band.
    In all bands above this one the plots start at the lower boundary and curve up before curving back down to asymptotically approach the lower boundary as observed in Fig. \textbf{CITE ALL Figures which have this feature.}


    $k_\perp = 0$ has no physical significance but it ends up critically controlling the shape of the plot in each band.
    When $\kappa_j > \frac{5}{2}$ the dispersion relation in the limit $k_\perp \rightarrow 0$ reduces to:
    %
    \begin{equation}
        \epsilon(0, \omega) = 1 - \sum_j \frac{4 (\kappa_j - \frac{3}{2}) v^2_{th}}{\lambda^2_{vcj}} \times \sum_{n = 1}^3 \alpha_n \frac{n}{2} \frac{1}{(\kappa_j - n + \frac{1}{2})} \frac{1}{(\omega^2 - \omega^2_{cj})}
    \end{equation}

    As $\omega$ increases from its lowest value of $\omega_{cj}$ this function starts increasing monotonically from a value of $-\infty$ asymptotically approaching a final value of 1.
    The continuity of the function, its monotonicity, and the Bolzano's Theorem guarantee that there exists a single root $\omega_0$.
    In the case of \textbf{CITE Henning} this single root can be calculated analytically: $\frac{\omega_0}{\omega_{cj}} = \frac{\omega_{UH}}{\omega_{cj}} = 5.099$.

    Crucially, $\epsilon(k_\perp, \omega)$ is continuous between the bands and the the $k_\perp = 0$ ($\omega$-axis).
    This means that in the band containing $\omega_0$ the plot has to start at $(0, \omega_0)$ since points on the plot correspond to roots of the function $\epsilon(k_\perp, \omega)$ which is continuous between the $\omega$-axis and the region of the band.
    Since $\epsilon(0, \omega < \omega_0) < 0$ while the function is positive at large $k_\perp$ continuity requires that the $k_\perp = 0$ segement must be disconnected from the large $k_\perp$ regime by the plot of roots which is why the plot starts from the top boundary and curves down to the lower boundary in this case.
    Similarly the fact that $\epsilon(0, \omega > \omega_0) > 0$ means that this segment must be connected with the large $k_\perp$ regime with no plot of roots in the middle.
    This is why the plot starts on the lower boundary and curves back to it leaving the segment and the right-most region connected.

    The second category of plots correspond to $\Lambda_j \neq 0$ and $\kappa_j < \frac{5}{2}$ for at least one of the species.
    In this case the plots in all of the bands start at some intermediate value (not at either of the boundaries) as observed in Fig. \textbf{CITE ALL Figures which have this feature}.
    It can be shown that for these values of $\Lambda_j$ and $\kappa_j$ the dispersion relation has negative powers of $2 \lambda_j$ which means $\epsilon(k_\perp = 0, \omega)$ is not finite.

    There is however a single exception in every band, a value of $\omega$ where $\kappa_j - n + \frac{5}{2} - \frac{\omega}{\omega_{cj}} = -p$ where $p \in \mathbb{N}$.
    At this value $\left| \Gamma (\kappa_j - n + \frac{5}{2} - \frac{\omega}{\omega_{cj}} + m) \right| \rightarrow \infty$ for the same values of $m$ for which the power of $2 \lambda_j$ is negative causing these terms to vanish and the dispersion relation to become finite.
    The Gamma function changes sign on either side of this exceptional value of $\omega$.

    Therefore, on the $k_\perp = 0$ axis, in each band, the dispersion relation starts at $-\infty$ at the bottom, becomes finite at a single point and then switches to $+\infty$ above it.
    Continuity now requires that the plot of roots touches the singular finite point where the function switches from $-\infty$ to $+\infty$.

    The continuity of the dispersion relation (except for $\frac{\omega}{\omega_{cj}} \in \mathbb{Z}^+$) allows us to understand the existence of bands in the graph, the shape of each plot in these bands, as well as the two broad categories of plot shapes.

\end{document}
